\documentclass[12pt, a4paper]{article}
\usepackage[utf8]{inputenc}
\usepackage{physics}
\usepackage[english]{babel}
\usepackage[margin=2cm]{geometry}
\usepackage{graphicx}
\usepackage{float} %use the option [H]
\graphicspath{ {images/} }
\usepackage{amsthm} %lets us use \begin{proof}
\usepackage{amsmath}
\DeclareMathOperator{\arcsinh}{arcsinh}
\usepackage{amssymb} %gives us the character
\usepackage{CJKutf8}
\usepackage[export]{adjustbox}
\setlength{\parindent}{0cm} % starting spaces
\setlength{\parskip}{1em} % paragraph width
\usepackage{array}
\usepackage{tabularx}
\usepackage{markdown}
\title{\textbf{Week-1}}
\author{王弘禹}
\date\today
\begin{document}
\begin{CJK*}{UTF8}{bsmi}
\linespread{1.5}
\maketitle
\newcommand{\st}[1]{\section*{#1}}
\newcommand{\sst}[1]{\subsection*{#1}}
\newcommand{\ssst}[1]{\subsubsection*{#1}}
\newcommand{\dsp}{\displaystyle}
\newcommand{\dx}{\,dx}
\newcommand{\dphi}{\,d\phi}
\newcommand{\dtheta}{\,d\theta}
\newcommand{\dy}{\,dy}
\newcommand{\dz}{\,dz}
\newcommand{\dr}{\,dr}
\newcommand{\drho}{\,d\rho}
\newcommand{\tb}{\textbf}
\st{1.5}
\sst{8.}
\begin{align}
    C_{ij} &= {(AB)}_{ij} - {(BA)}_{ij}\\
    &= \sum_r (A_{ir}B_{rj} - B_{ir}A_{rj})\\
    \sum_i C_{ii} &= \sum_i\sum_r (A_{ir}B_{ri} - B_{ir}A_{ri})\\
    &= \sum_{(i,r),\, i\le r}(A_{ir}B_{ri}-B_{ri}A_{ir})=0
\end{align}
\st{1.6}
\sst{9.}
Assume \(A_{kk}=0\) and \(A_{ii}\neq 0,\,i\neq k\), then, we can reduce A to the form below.
\begin{align}
    A^\prime_{ij} = \delta_{ij},\, j\neq k
\end{align}
In this form, the linear combination of rows can't be the k-th row of identity matrix.
\sst{11.}
Trivial problem. We can first do row operation to make \(PA\) be a row-reduced echelon, and then do column operations to make \(PAQ\) be column-reduced echelon matrix.
\end{CJK*}
\end{document}