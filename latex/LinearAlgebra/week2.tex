\documentclass[12pt, a4paper]{article}
\usepackage[utf8]{inputenc}
\usepackage{physics}
\usepackage[english]{babel}
\usepackage[margin=2cm]{geometry}
\usepackage{mathrsfs}  
\usepackage{graphicx}
\usepackage{float} %use the option [H]
\graphicspath{ {images/} }
\usepackage{amsthm} %lets us use \begin{proof}
\usepackage{amsmath}
\DeclareMathOperator{\arcsinh}{arcsinh}
\usepackage{amssymb} %gives us the character
\usepackage{CJKutf8}
\usepackage[export]{adjustbox}
\setlength{\parindent}{0cm} % starting spaces
\setlength{\parskip}{1em} % paragraph width
\usepackage{array}
\usepackage{tabularx}
\usepackage{markdown}
\title{\textbf{Week-2}}
\author{王弘禹}
\date\today
\begin{document}
\begin{CJK*}{UTF8}{bsmi}
\linespread{1.5}
\maketitle
\newcommand{\st}[1]{\section*{#1}}
\newcommand{\sst}[1]{\subsection*{#1}}
\newcommand{\ssst}[1]{\subsubsection*{#1}}
\newcommand{\dsp}{\displaystyle}
\newcommand{\dx}{\,dx}
\newcommand{\dphi}{\,d\phi}
\newcommand{\dtheta}{\,d\theta}
\newcommand{\dy}{\,dy}
\newcommand{\dz}{\,dz}
\newcommand{\dr}{\,dr}
\newcommand{\drho}{\,d\rho}
\newcommand{\tb}{\textbf}

\st{2.3}
\sst{11.}
\ssst{(a)}
Consider two vectors \(v_1,v_2\) in V
\begin{align}
    &(cv_1+v_2)\in V,\quad c\in R\\
    \implies & \text{V is a vector space over real numbers.}
\end{align}
\ssst{(b)}
There are six components in the basis:\begin{align}
    &\beta_1 = \begin{bmatrix}
        1 & 0\\
        0 & -1
    \end{bmatrix} &\beta_2 = \begin{bmatrix}
        i & 0\\
        0 & -i
    \end{bmatrix}\\
    &\beta_3 = \begin{bmatrix}
        0 & 1\\
        0 & 0
    \end{bmatrix} &\beta_4 = \begin{bmatrix}
        0 & i \\
        0 & 0
    \end{bmatrix}\\
    &\beta_5 = \begin{bmatrix}
        0 & 0\\
        1 & 0 
    \end{bmatrix} &\beta_6 = \begin{bmatrix}
        0 & 0\\
        i & 0
    \end{bmatrix}
\end{align}
\ssst{(c)}
The proof that W is a subspace of V is same method like (a). There are four components in the basis:
\begin{align}
    &\beta_1 = \begin{bmatrix}
        1 & 0\\
        0 & -1
    \end{bmatrix} &\beta_2 = \begin{bmatrix}
        i & 0\\
        0 & -i
    \end{bmatrix}\\
    &\beta_3 = \begin{bmatrix}
        0 & 1\\
        -1 & 0
    \end{bmatrix} &\beta_4 = \begin{bmatrix}
        0 & i \\
        i & 0
    \end{bmatrix}
\end{align}
\end{CJK*}
\end{document}