\documentclass[12pt, a4paper]{article}
\usepackage[utf8]{inputenc}
\usepackage{physics}
\usepackage[english]{babel}
\usepackage[margin=2cm]{geometry}
\usepackage{graphicx}
\usepackage{float} %use the option [H]
\graphicspath{ {images/} }
\usepackage{amsthm} %lets us use \begin{proof}
\usepackage{amsmath}
\DeclareMathOperator{\arcsinh}{arcsinh}
\usepackage{amssymb} %gives us the character
\usepackage{CJKutf8}
\usepackage[export]{adjustbox}
\usepackage{hyperref}
\setlength{\parindent}{0cm} % starting spaces
\setlength{\parskip}{1em} % paragraph width
\usepackage{array}
\usepackage{tabularx}
\usepackage{markdown}
\title{\textbf{The derivation of black body radiation}}
\author{王弘禹}
\date\today
\begin{document}
\begin{CJK*}{UTF8}{bsmi}
\linespread{1.5}
\maketitle
\newcommand{\st}[1]{\section*{#1}}
\newcommand{\sst}[1]{\subsection*{#1}}
\newcommand{\ssst}[1]{\subsubsection*{#1}}
\newcommand{\dsp}{\displaystyle}
\newcommand{\dx}{\,dx}
\newcommand{\dphi}{\,d\phi}
\newcommand{\dtheta}{\,d\theta}
\newcommand{\dy}{\,dy}
\newcommand{\dz}{\,dz}
\newcommand{\dr}{\,dr}
\newcommand{\drho}{\,d\rho}
\newcommand{\tb}{\textbf}

\st{主題}
討論實驗十九、光譜分析的【思考問題】為何黑體輻射所產生的是連續光譜?以前高中就學過是普朗克引入能量量子化才解決黑體輻射、紫外災變,但其實我撰寫此報告前也不知道普朗克具體是怎麼做的。因此,為回答該思考問題,本報告將推導一次黑體輻射的公式。
\st{Black body and wave in a box}
Consider a box with perfectly conducting walls, and a electromagnetic wave inside it. Since the electric field \emph{must be zero at the walls of the box}, we can expect that only the standing wave can fit this boundary condition.

Assume it's a cubic box with length \(L\), thus, the modes of oscillation can be written
\begin{equation}
    R(x,y,z) = R_0 \sin{(k_x x)}\sin{(k_y y)}\sin{(k_z z)}
\end{equation}
and $l,m,n \in \mathbb{N}$
\begin{align}
    k_x =\dfrac{\pi l}{L}, \, k_y = \dfrac{\pi m }{L},\, k_z = \dfrac{\pi n}{L}
\end{align}
Applied wave equation
\begin{equation}
    \nabla^2 A  = \dfrac{1}{c^2}\dfrac{\partial^2 A}{\partial t^2}\implies {k_x}^2+{k_y}^2+{k_z}^2=\dfrac{\pi^2}{L^2}(l^2+m^2+n^2)= \dfrac{\omega^2}{c^2}
\end{equation}
The number of modes of oscillation is equal to the number of latice points in l-m-n coordinate. Let the number of modes denote as \(N_T\) and a postive integer M
\begin{align}
    l,m,n \leq M \implies N_T = l\cdot m\cdot n \;(\text{combination num.})= V_{lmn}\;(\text{Volume})
\end{align}
Let 
\begin{align}
    &p^2 = l^2 + m^2 + n^2 \xrightarrow{} p^2 =\dfrac{L^2}{\pi^2}\dfrac{\omega^2}{c^2}\\
    &k^2 = {k_x}^2 +{k_y}^2 + {k_z}^2 \xrightarrow{} k=\dfrac{\pi p}{L}\\
    &d(N_T) = N(p)\,dp = N(k)\,dk
\end{align}
And derive $N(k)$
\begin{align}
    d(N_T) = dV = \dfrac{1}{8}(4\pi p^2)\,dp = N(p)\,dp \implies N(k)\,dk = \dfrac{L^3}{2\pi^2}k^2\,dk
\end{align}
Denote \(V = L^3\) and \(\tb k = \langle k_x,k_y,k_z\rangle\)
\begin{align}
    &\mid\tb k \mid=k = \dfrac{2\pi}{\lambda} = \dfrac{2\pi}{c}\nu\\
    &N(k)\,dk = N(\nu)\,d\nu = \dfrac{4\pi V}{c^3} \nu^2 \,d\nu
\end{align}
Since the polarisation of electromagnetic wave,  the number of states \emph{per unit volume} is
\begin{equation}
    dN_T = 2\cdot \dfrac{4\pi\nu^2}{c^3}\,d\nu= \dfrac{8\pi\nu^2}{c^3}\,d\nu
\end{equation}
\st{Ultraviolet Catastrophe}
Denote the energy density be \(u(\nu)\) and average energy to each mode of oscillation \(\overline{E}\)
\begin{align}
    u(\nu) = \dfrac{8\pi \nu^2}{c^3}\overline{E}
\end{align}
Because the average energy of a harmonic oscillator \(\overline{E}=kT\)
\begin{equation}
    u(\nu) = \dfrac{8\pi\nu^2kT}{c^3}
\end{equation}
It results \emph{ultraviolet catastrophe}, the total energy diverges
\begin{equation}
    \int_0^\infty u(\nu)d\nu =\int_0^\infty \dfrac{8\pi\nu^2kT}{c^3}d\nu \xrightarrow{} \infty
\end{equation}
\st{Solution of the catastrophe}
In above section, the waves are constrained to fit into the box. Now we have a further constraint. The energy of the mode is \(E(\nu)=nh\nu\), where \(h\) is planck constant and \(n\) is postive number.
\sst{Boltzmann distribution}
The probability that a single mode has energy \(E_n = nh\nu\)
\begin{equation}
    p(n) = \dfrac{\exp(-E_n/kT)}{\sum\limits_{n=0}^\infty \exp(-E_n/kT)}
\end{equation}
The mean energy become a function of \(\nu\)
\begin{align}
    \overline{E_\nu} = \sum\limits_{n=0}^\infty E_n p(n)=\dfrac{\sum\limits_{n=0}^\infty nh\nu \exp(-nh\nu/kT)}{\sum\limits_{n=0}^\infty \exp(-nh\nu/kT)}
\end{align}
It looks like power series, let us substitute \(x=\exp(-hv/kT)\).
\begin{align}
    \overline{E_\nu} =  h\nu \dfrac{\sum\limits_{n=0}^\infty nx^n}{\sum\limits_{n=0}^\infty x^n} = h\nu x\dfrac{(1+2x+3x^2+\ldots)}{(1+x+x^2+\ldots)}
\end{align}
Since \(\dfrac{1}{1-x} = 1+x+x^2+\ldots\) and \(\dfrac{1}{{(1-x)}^2}=1+2x+3x^2+\ldots\)
\begin{equation}
    \overline{E_\nu} = h\nu\dfrac{x}{1-x} =  \dfrac{h\nu}{\exp(h\nu/kT)-1}
\end{equation}
Remenber (13) tell us \(u(\nu) = \dfrac{8\pi\nu^2}{c^3}\overline{E}\), then we get \emph{Planck distribution function}
\begin{equation}
    u(\nu)d\nu = \dfrac{8\pi\nu^2}{c^3}\overline{E_\nu}= \dfrac{8\pi h\nu^3}{c^3}\dfrac{1}{\exp(h\nu/kT)-1}d\nu
\end{equation}
where \(u(v)\) is energy density. Change the variable to \(\lambda\)
\begin{equation}
    u(\nu)d\nu = u(\lambda)d\lambda = -\dfrac{8\pi h}{\lambda^3}\dfrac{1}{\exp(hc/kT\lambda)-1}(\dfrac{h}{\lambda^2})\,d\lambda
\end{equation}
Draw the graph \(\mid u(\lambda)\mid -\lambda\)
\begin{figure}[H]
    \begin{center}
        \includegraphics[width=0.7\textwidth]{"Screenshot 2022-06-13 120427.png"}
    \end{center}
\end{figure}
\st{後記}
會用英文寫是因為我發現我不太會用中文寫,畢竟參考資料本來就是英文,要把一些敘述精確的轉成中文反而好難。

主要參考的資料是文章\href{https://edisciplinas.usp.br/pluginfile.php/48089/course/section/16461/qsp_chapter10-plank.pdf}{"The Derivation of the Planck Formula"}以及另一個系上的強者同學。

由於推導過程都是現有的知識,我僅僅做的是將其學會後,用自己的理解重新編排推導過程罷了,感謝閱讀。
\end{CJK*}
\end{document}