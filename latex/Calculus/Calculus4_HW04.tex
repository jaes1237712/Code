\documentclass[12pt, a4paper]{article}
\usepackage[utf8]{inputenc}
\usepackage{physics}
\usepackage[english]{babel}
\usepackage[margin=2cm]{geometry}
\usepackage{graphicx}
\usepackage{float} %use the option [H]
\graphicspath{ {images/} }
\usepackage{amsthm} %lets us use \begin{proof}
\usepackage{amsmath}
\DeclareMathOperator{\arcsinh}{arcsinh}
\usepackage{amssymb} %gives us the character
\usepackage{CJKutf8}
\usepackage[export]{adjustbox}
\setlength{\parindent}{0cm} % starting spaces
\setlength{\parskip}{1em} % paragraph width
\usepackage{array}
\usepackage{tabularx}
\usepackage{markdown}
\title{\textbf{Calculus HW4}}
\author{王弘禹}
\date\today
\begin{document}
\begin{CJK*}{UTF8}{bsmi}
\linespread{1.5}
\maketitle
\newcommand{\st}[1]{\section*{#1}}
\newcommand{\sst}[1]{\subsection*{#1}}
\newcommand{\ssst}[1]{\subsubsection*{#1}}
\newcommand{\dsp}{\displaystyle}
\newcommand{\dx}{\,dx}
\newcommand{\dphi}{\,d\phi}
\newcommand{\dtheta}{\,d\theta}
\newcommand{\dy}{\,dy}
\newcommand{\dz}{\,dz}
\newcommand{\dr}{\,dr}
\newcommand{\drho}{\,d\rho}
\newcommand{\tb}{\textbf}
\st{16 Review}
\sst{36.}
考慮一個單位球面$\,U$ 與題目給的橢圓球面 $E$形成的區域$\Omega$
\begin{align}
    &\iiint\limits_\Omega \div{\tb F} = \iint\limits_E \tb F\cdot \tb n +\iint\limits_U \tb F\cdot\tb n \\
    &\iint\limits_E \tb F\cdot \tb n = 0-(4\pi)
\end{align}

\sst{39.}
\begin{align}
    \div{\tb F} = 3
\end{align}
由於是常數,只需算剩下的部分體積為多少就好。
\begin{align}
    V = 8-1 \implies \iiint\limits_E \div{\tb F}\,dV = 21
\end{align}
\st{11.1}
\sst{39.}
\ssst{Claim}
Suppose that $\dsp\lim_{n\to \infty}\sin n =L$
跟
\end{CJK*}
\end{document}